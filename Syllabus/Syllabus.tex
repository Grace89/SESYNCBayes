\documentclass[12pt, oneside]{article}
\usepackage[margin=1in]{geometry}                		
\geometry{letterpaper}   
\newcommand\NoIndent[1]{%
  \par\vbox{\parbox[t]{\linewidth}{#1}}%
}    
\usepackage{enumitem}            	
\usepackage{graphicx}
\usepackage{wrapfig}
\usepackage{tabularx}
\usepackage{array}
\usepackage{ragged2e}
\usepackage{hyperref}
\providecommand{\LyX}{L\kern-.1667em\lower.25em\hbox{Y}\kern-.125emX\@}
\usepackage{fancyheadings}
\pagestyle{fancy}
\lhead {SESYNC Bayesian Short Course}
\rhead{August 19-28, 2015}

\title{Short Course: Bayesian Modeling for \\
Ecologists and Social Scientists}
\author{N. Thompson Hobbs, Mary B. Collins, and Christian Che-Castaldo}
\date{August 1-10, 2016}	
\begin{document}
\maketitle

\section*{Course preparation}
\begin{itemize}
\item Consider studying chapters 1-5 in ``Bayesian Models: A Statistical Primer for Ecologists," by Hobbs and Hooten, in preparation for the course.  We will cover these materials at the start of the course.
%\item Did we decide to ask participants to come up with analysis ideas?
\end{itemize}

\section*{General logistics}
\begin{itemize}
\item{Each day the course will begin at 9am and end at 5pm.}
\item{Coffee will be provided throughout the day by SESYNC.}
\item{Each day, SESYNC will provide various snacks around 10:30am and 3:30pm. Some days we will plan to break, other days you can feel free to grab a snack during our lab activities if you need it.}
\item{Each day we will break from 12:30 to 1:30 for lunch, which will be provided by SESYNC.}
%\item{SESYNC has set our course up with a listserve, feel free to use it. \\
%bayescourse@lists.sesync.org}
%\item{SESYNC has set our course up with a collaborative website. This is where you will find all of the course materials including, lecture slides, lab materials, and suggesting readings.  The site address is \href{http://sesync.us/bayescouse}{http://sesync.us/bayescouse}.  You should have already received login instructions from the SESYNC compute staff.}  
\item{Course materials can be printed at SESYNC if desired.  There is a computer with printer access in the common area.}
\end{itemize}
\pagebreak
\noindent


\subsection*{Day 1: What Sets Bayes' Apart} (Mon Aug 1, 2016) \\
\textbf{Reading}: Hobbs and Hooten---Preface and Chapter 1-2\\
Introductions, compute overview (30 min total) \\
\textbf{Lectures}: \\
(1) Broad course overview (\emph{Hobbs})\\
(2) What sets Bayes' apart?  (\emph{Hobbs})\\
(3) Laws of probability (\emph{Collins})\\
\textbf{Labs}: \\
(1) Confidence intervals.\\
(2) DAGs and Marginal Distributions\\
\emph{Happy Hour at SESYNC}


\subsection*{Day 2: Introduction to Probability and Maximum Likelihood} (Tues Aug 2, 2016)\\
\textbf{Reading}: Hobbs and Hooten, Chapter 3\\
\textbf{Lectures}: \\
(1) Probability distributions  (\emph{Hobbs})\\
(2) Maximum likelihood (\emph{Collins})\\
\textbf{Labs}: \\
(1) Distributions lab\\
(2) Light limitation of trees\\	

\subsection*{Day 3: Bayes' Derived and Intro to Priors} (Wed Aug 3, 2016)\\
\textbf{Reading}: Hobbs and Hooten---Chapter 4-5\\
\textbf{Lectures}: \\
(1) Bayes' made simple: Deriving Bayes' theorem (\emph{Hobbs})\\
(2) Priors distributions I (\emph{Collins}) \\
\textbf{Labs}: \\
(1) Write R code for each component for Bayes' Law \\	
(2) Conjugate priors lab\\

\newpage
\subsection*{Day 4:  MCMC} (Thurs Aug 4, 2016)\\
\textbf{Reading}: Hobbs and Hooten---Chapter 7-8\\
\textbf{Lectures}: \\
(1) MCMC I: Overview (\emph{Hobbs})\\
(2) MCMC II: Implementing accept-reject sampling (\emph{Hobbs})\\
\textbf{Labs}: \\
(1) Gibbs sampling \\
(2) Accept-reject sampling \\

\subsection*{Day 5: JAGs} (Fri Aug 5, 2016)\\
\textbf{Lectures}: \\
(1) Simple Bayesian Regression (\emph{Che-Castaldo}) \\
(2) Inference from a single model (\emph{Hobbs})\\
\textbf{Labs}: \\
(1) JAGS primer\\
(2) Islands problem (including breakout regarding consequences of priors on non-linear derived quantities)\\

\subsection*{Day 6: Bayesian Models I} (Sat Aug 6, 2016)\\
\textbf{Reading}: Hobbs and Hooten---Chapter 6\\
\textbf{Lectures}: \\
(1) Hierarchical Models: Group-Level Effects (\emph{Hobbs}) including focus on \emph{vaguely} informative priors.\\
(2) Overview of Bayesian Hierarchical Models (\emph{Hobbs}) \\
\textbf{Labs}: \\
(1) Multi-level model on N\textsubscript{2}O \\
(2) Commonly encountered problems in analysis\\

\subsection*{Day 7: OFF} (Sun Aug 7, 2016)\\
 \newpage
\subsection*{Day 8: Bayesian Models II} (Mon Aug 8, 2016)\\
\textbf{Reading}: Hobbs and Hooten---Chapter 6\\
\textbf{Lectures}: \\
(1) Hierarchical models: Designed Experiments (\emph{Che-Castaldo}) \\
(2) Hierarchical models: Mixture, zero-inflation and occupancy (\emph{Hobbs})\\
\textbf{Labs}: \\
(1) Designed experiments \\
(2) Swiss birds I: Occupancy modeling\\

\subsection*{Day 9: Bayesian Models III} (Tues Aug 9, 2016)\\
\textbf{Reading}: Hobbs and Hooten---Chapter 8: Sections 1 and 5; Chapter 9\\
\textbf{Lectures}: \\
(1) Model checking (\emph{Hobbs})\\
(2) Model selection (\emph{Hobbs})\\
(3) Meta analysis including use of informed priors (\emph{Hobbs}) \\
\textbf{Labs}: \\
(1) N\textsubscript{2}O, part II: Model evaluation\\
(2) Model selection\\
(3) Meta Analysis \\

\subsection*{Day 10: Dynamic models} (Wed Aug 10, 2016)\\
\textbf{Lectures}: \\
(1) Hierarchical Models: Dynamic models (\emph{Hobbs})\\
\textbf{Labs}: \\
(1) Lynx: Dynamic models\\


\end{document}
